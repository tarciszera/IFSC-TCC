\chapter{Introdução}

Para o desenvolvimento de um transdutor é necessária sua caracterização, ou seja, dadas excitações de natureza especifica, verificar a reação do transdutor a essas excitações. Assim, deve-se ter um ambiente de testes que forneça algum tipo de excitação conhecida, permitindo medidas precisas.

No caso de transdutores magnéticos, além da influência do meio (Permeabilidade magnética), há um campo terrestre que não é constante e que deve ser compensado.

Na empresa Silicon Austria Labs (SAL), onde foi feito este projeto, foi desenvolvido um transdutor magnético. Este transdutor necessitava ser caracterizado e para isso um equipamento que gerasse um campo magnético homogêneo e estático deveria ser construído.

Este trabalho descreve todo o projeto e construção deste equipamento que é baseado em bobinas de Helmholtz, uma bobina para cada direção do espaço (x, y, z), tendo como resultado uma bobina de Helmholtz 3D. O controle das bobinas e tratamento de dados também é desenvolvido nesse projeto.

\section{Objetivos}

\subsection{Objetivo geral}

O objetivo do projeto é desenvolver um equipamento que possa gerar um campo magnético homogêneo capaz de compensar o campo magnético da terra, sendo empregado para análise de sensores magnéticos. 
Os requisitos foram retirados da experiência do orientador do projeto com projetos anteriores de objetivo parecido. A seguir os requisitos para o campo magnético gerado pelas bobinas:    

\begin{itemize}
    \item Intensidade de indução magnética entre 1 $\mu$T e 1,25 mT em uma bobina principal, sendo as outras duas numa faixa de 1 $\mu$T e 0,25 mT
    \item Passo de indução de 1 $\mu$T ou menos.
    \item Diferença de intensidade de indução magnética de qualquer parte da região sob controle menor que 1\% da intensidade de campo no centro da região.
    \item Diferença do ângulo entre os vetores da região e o vetor central menor que um grau.
\end{itemize}

Todos esses requisitos devem ser cumpridos numa região grande o suficiente para o transdutor inteiro a ser caracterizado. No caso deste projeto, é definida uma região de 20 mm³, que é maior do que o transdutor desenvolvido pela SAL. O campo deve manter-se constante do começo ao fim de cada medida realizada.

\subsection{Objetivos específicos}

\begin{itemize}
    \item Projetar as bobinas de Helmholtz necessárias para gerar um campo tridimensional com o auxílio de um simulador.
    \item Projetar o suporte das bobinas de Helmholtz e todas as estruturas mecânicas relacionadas com o posicionamento dos dispositivos no sistema.
    \item Confeccionar as bobinas de Helmholtz e as peças do sistema.
    \item Projetar um sistema para controlar o campo gerado pelas bobinas de Helmholtz.
\end{itemize}
